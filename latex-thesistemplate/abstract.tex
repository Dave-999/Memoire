% abstract.tex (Abstract)

\addcontentsline{toc}{chapter}{Abstract}

\chapter*{Abstract}
Solving systems of linear equations is a common computational problem well
known to mathematicians, scientists and engineers.  Several algorithms exist 
for solving this problem.  However, when the equations contain {\it interval 
coefficients\/} (i.e., intervals in which the desired coefficient values are 
known to lie), the problem may not be solvable in any reasonable sense.  In 
fact, it has been shown that the general problem of solving systems of linear 
equations with interval coefficients is NP-{\it hard}, i.e., extremely difficult
and (it is believed) unsolvable; thus, no feasible algorithm can ever be 
developed that will solve all particular cases of this problem.

It turns out, though, that the widths of the interval coefficients are quite 
small in a large number of the linear systems having interval coefficients.  
This becomes readily apparent when we learn that the intervals typically come 
from measurements.

Any measurement of a physical quantity is limited
by the precision and accuracy of the measuring device.  To be of practical
use, the measuring devices used in science and industry must be
reasonably accurate.  This implies that, for the most part, the actual values
associated with measurements lie within relatively narrow intervals.  Indeed,
manufacturers often guarantee the error of their instruments to be very small.

Thus, we desire to look only at {\it narrow-interval\/} coefficients when 
considering the development of an algorithm for solving linear systems with 
interval coefficients.  As there already exists an algorithm that solves 
most such systems, developing such an algorithm seems indeed promising.
Therefore, the goal of this thesis is to answer the following question:

\begin{quote}
{\it Can a feasible algorithm be developed for the general problem of
solving systems of linear equations with narrow-interval coefficients?}
\end{quote}

\noindent
We show here that this problem, that of solving systems of linear 
equations with narrow-interval coefficients, is NP-hard; thus, we do not
consider it possible to develop a feasible algorithm that will solve all 
particular cases of this problem.

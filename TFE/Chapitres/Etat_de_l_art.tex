\chapter{State of the art}
\label{Chap2}
Parler de l'AlSi10Mg; quel est l'interet de travailler avec? Difficultés? (reflectivité etc)\\

Microstructure homogène, diagramme de phase\\

Fonctionnement du SLM\\

The properties of parts produced trough selective laser melting (SLM) stem from the coupled effects of a great deal of parameters (see figure \ref{fig:param} ) \parencite{Aboulkair140820}.  %Les caractérisitiques des pièces produites grâce à la fusion laser sélective (SLM) sont le fruit de l'action simultanée et couplée d'un grand nombres de paramètres (see figure \ref{fig:param} ) \parencite{Aboulkair140820}.
Results are very sensitive to their variations. The process parameters must thus be monitored thoroughly. This complicates the search for their optimisation, still not fully resolved for aluminium alloys.\\%.Les résultats sont très sensibles à leurs variations et il est donc nécessaire de les contrôler méticuleusement. Pour ces raisons, il n'est pas simple d'étudier leurs impacts.\\
\begin{figure}[th]
\centering
\includegraphics[scale=0.42]{Images/Param}
\decoRule

\caption[Parameters involved in SLM]{Parameters involved in SLM (from Aboulkhair et al, 2014)}
\label{fig:param}
\end{figure}

In recent years, works aiming at facing this challenge multiplied. The minimisation of the porosity is at the center of attention. It is indeed closely related to the quality of the mechanical properties. As porosity contributes to lowering the load-bearing surface, it reduces the apparent material strength. It was also observed to have a critical influence on the fatigue life of the produced parts. Their lifetime is especially diminished if the values of pores amount and size go beyond a certain threshold \parencite{Brandl121509}. Studies investigating the effects of various parameters on the AlSi10Mg fabrication trough SLM abound in the literature.\\

The analysis of the paired impacts of the laser power P and scan speed $v_s$ provides a first insight. As depicted by figures \ref{fig:Pvs} and \ref{fig:Pvs2}, low P and high $v_s$ lead to an insufficient energy input to melt the powder and re-melt the substrate, which causes the formation of droplets \parencite{Kempen110817} . The opposite leads to good penetration but also to distortions and irregularities.   A trend to use both high P and $v_s$ rose in accordance with these findings. Doing so as the advantage to increase productivity. However, it also has multiple downsides including a decrease of the surface quality due to balling, excessive spatter, and an augmented gas induced porosity \parencite{Mertens170406}. Therefore, a trade-off must be found. \\

A popular approach is to regroup multiple operating parameters into one, the volumetric energy density E. It is estimated trough the following formula: 
$$E_d=\frac{P}{v_s h t} $$
where t is the layer thickness and h is the hatch space. As a rule of thumb, $E_d$ should be chosen in the range between 60 and 75 [$\frac{J}{mm^3}$] \parencite{Read150417}. However, the criterion is insufficient and others should be considered such as melt pools overlapping \parencite{Tang170309}. Almost no studies were carried out to optimize h and t independently. Their values lie generally respectively in the intervals [20 ; 60] [$\mu m$] and [50 ; 200] [$\mu m$].\\

\begin{figure}[th]
\centering
\includegraphics[scale=0.32]{Images/Pvs}
\decoRule
\caption[Process window for SLM of AlSi10Mg, based on the top view of single track scans]{Process window for SLM of AlSi10Mg, based on the top view of single track scans (from Kempen et al, 2011)}
\label{fig:Pvs}
\end{figure}

\begin{figure}[th]
\centering
\includegraphics[scale=0.32]{Images/Pvs2}
\decoRule
\caption[Process window for SLM of AlSi10Mg, based on the front view of single track scans]{Process window for SLM of AlSi10Mg, based on the front view of single track scans (from Kempen et al, 2011)}
\label{fig:Pvs2}
\end{figure}

The other process parameters will be covered for the sake of completeness. Let us 
first look into the particle-related parameters. The particle size $D_a$ of the powder should be as small as possible to ensure a good flowability and allow for thin layers \parencite{Kempen110817}. Typical values stretch from 15 to 60 [$\mu m$]. The size distribution is more delicate to outline. On one hand, wider distributions often generate better bed density and parts with higher density and better surface finish. On the other hand, narrower ones usually provide better flowability and parts with better strength and hardness \parencite{Liu1101}. In most cases, a middle ground between the two should be sought. In SLM applications, powder is often successively recycled multiple times. This leads to their progressive contamination with moisture, which causes an increase of hydrogen porosity in the produced parts \parencite{Weingarten151102}. The problem can be overcome by drying the powder or using fresh one. Unfortunately - in the case of aluminium alloys - no findings were made regarding the prediction of a threshold at which measures should be taken \parencite{aboulkhair2017}.    \\

Second, the choice of scan pattern is also of great importance. .. Building direction...\\

Other laser-related parameters - the spot size and the pulse properties - can also be tuned. Only the laser spot size at the 99\% contour $\phi_{99\%}$ is frequently cited in literature. Its value lies between 100 and 200 $\mu m$.\\

Finally, pressure and temperature..\\

Comparer les résultats avec alliage coulé/forgé\\

Once the porosity problem is sorted out, other matters can be addressed such as productivity and surface roughness. The latter is problematic as the surface finish obtained with SLM is typically of such poor quality that all cracks initiate near the surface for a sample with apparent relative density $\rho_{rel}>99\%$ \parencite{Brandl121509}. Polir ou changement paramètres fab.\\ %Durant les dernières années, les travaux visant à optimiser les conditions de fabrication se sont multipliés. La minimisation de la porosité est au centre de l'attention: elle est en effet liée à la qualité des propriétés mécaniques. De plus au delà d'un seuil, des risques de rupture prématurée peuvent apparaitre (source). (Initiation de sites de propag ... ) On s'intéresse ensuite à optimiser d'autres caractéristiques du matériau et à la productivité de la technique.

Post-traitements dont traitements thermiques, sur lesquels on se focalise. Expliquer\\

%Recherches biblio....
%\section{Comment référencer?}
%The \code{biblatex} package is used to format the bibliography and inserts references such as this one \parencite{Reference1}. The options used in the \file{main.tex} file mean that the in-text citations of references are formatted with the author(s) listed with the date of the publication. Multiple references are separated by semicolons (e.g. \parencite{Reference2, Reference1}) and references with more than three authors only show the first author with \emph{et al.} indicating there are more authors (e.g. \parencite{Reference3}). This is done automatically for you. %To see how you use references, have a look at the \file{Chapter1.tex} source file. Many reference managers allow you to simply drag the reference into the document as you type.
%
%Scientific references should come \emph{before} the punctuation mark if there is one (such as a comma or period). The same goes for footnotes\footnote{Such as this footnote, here down at the bottom of the page.}. You can change this but the most important thing is to keep the convention consistent throughout the thesis. Footnotes themselves should be full, descriptive sentences (beginning with a capital letter and ending with a full stop). The APA6 states: \enquote{Footnote numbers should be superscripted, [...], following any punctuation mark except a dash.} The Chicago manual of style states: \enquote{A note number should be placed at the end of a sentence or clause. The number follows any punctuation mark except the dash, which it precedes. It follows a closing parenthesis.}
%
%The bibliography is typeset with references listed in alphabetical order by the first author's last name. This is similar to the APA referencing style. To see how \LaTeX{} typesets the bibliography, have a look at the very end of this document (or just click on the reference number links in in-text citations).
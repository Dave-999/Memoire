\chapter{Introduction}
\label{Chap1}

In highly demanding industries such as aerospace and automotive, cutting edge innovation is a key factor to stay competitive and profitable. Ever more efficient materials must constantly be developed.\\

Over the last decade, a very promising technology has emerged: selective laser melting (SLM). It has many appeals compared to other manufacturing processes, such as higher adaptability, better reliability and lower cost. However, a great deal of parameters influence the properties of the manufactured parts. Therefore, substantial research is needed to optimise the process. Studies must be done independently for each material, as the findings for one usually do not transpose to others.\\

Using SLM with AlSi10Mg and other aluminium alloys has gained in popularity in recent years. The material has several fields of applications due to its light weight and relatively high strength.\\

Although great progress was made in additive manufacturing of AlSi10Mg, some issues remain unsolved. The ductility and strength of the manufactured parts vary considerably in literature. The causes are not always clearly understood. The inconsistency of the mechanical properties puts a brake on the commercial use of the technique.\\

Insight will be sought in this thesis by investigating the residual stresses induced by the manufacturing technique. First, literature will be reviewed. The process parameters will then be optimised so as to maximise density. The effect of heat treatments will finally be discussed, both in terms of mechanical properties and microstructure. A monitoring of the AlSi10Mg powder composition and grain sizes as functions of the recycling conditions will also be conducted.\\
\chapter{Discussion}
\label{Chap5}
Que conclure d'après les résultats?

\section{Parameters optimisation}

\section{Reproducibility}

\subsection{Melt pool size distribution}


\section{Powder ageing}
\subsection{Grain size and distribution}


\subsection{Composition}

\section{Density measures assessments}
\subsection{Measures comparison}
%\subsection{Hydrostatic weighing}

One can expect the first option to reduce the risks of air trapping by the surface roughness during the underwater weighing, which can distort the results (by overestimating the closed porosities volume). %to observe potential inhomogeneous distributions of the closed porosities in the specimens.

\subsection{Relative optical density image analysis}
The estimation of the relative density trough RODIA can be distorted on many grounds. First, the distribution of porosities is inhomogeneous on the analysed surface. Multiple photos must thus be taken with a systematic manner for each specimen to constitute a representative sample.\\

Second, the quality of the photographs has a critical role. The isolation of the porosities during the thresholding requires a substantial difference of pixel intensity between the holes and the material. Since some porosities and some zones of the material can appear respectively brighter or darker than was is expected, there are risks that one isolates spots and/or not actual porosities. Additionally, the thresholding is manual and thus prone to slight human errors. \\

Most importantly, the finite resolution of the camera implies that sufficiently small porosities are not visible on the pictures. Results confirmed that pictures with lower resolution had the tendency to lead to the overestimation of the relative density (see section \ref{RRODIA}). The method is thus presumably positively biased. However, the observed effect is minor: this is probably due to the fact that the undetected porosities are the smallest, which influence the less the calculated density value.\\

Taking pictures at refined magnification could be considered to better the precision of the method. This would, however, require to augment the number of analysed pictures to have a sample of pictures as representative. A picture with doubled magnification covers indeed four times less surface. The number of analysis should thus be quadrupled to take as much information into account.\\


\section{Heat treatments}

\subsection{Heating process??}

\subsection{Microstructure}

\subsection{Mechanical properties}

\subsection{Residual stress}

While the usual stress-relief treatment for aluminium alloys -to hours holding at 300$^\circ$ C- does indeed relieve stresses inside the specimen, it also triggers significantly the diffusion of alloying elements, altering the material microstructure.

\subsection{Optimisation}



%\section{Mechanical testing}
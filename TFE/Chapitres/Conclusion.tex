\chapter{Conclusion}
\label{Chap6}
%\textcolor{gray}{They incorporate in a synthetic way the main results and compare them with the
%initial objectives. Generally, this final chapter also presents prospects for the continuation of the
%work undertaken.}
In this master thesis, an analysis of the impact of stress relief heat treatments was carried out. Results for the residual stresses measurements were unsatisfactory: they suggested that the residual stresses of heat treated samples were equivalent to or bigger than the ones of as-built samples. Possible explanations were provided.\\

Nevertheless, multiple instructive observations were made. The importance of powder monitoring was first highlighted. The assessment of the relative density measurements was then conducted. Results indicated that hydrostatic weighing (after polishing) and relative optical density image analysis methods could constitute bound values for the actual relative density value.\\

Optimal process parameters were identified with the aim of achieving high relative density. The reproducibility was addressed, both for the optimal parameters and for suboptimal ones. The former led to relative density values ranging from 99.4\% to 99.9\%. High hardness values were obtained compared to literature. The underlying reasons were identified.\\

Plausible hypotheses were made to explain the mechanical properties for as-built and heat-treated specimens. The major role of residual stresses was emphasised. The use of a pre-heating plate during the manufacturing process seems to substantially improve the tensile properties. An empirical relationship linking the ultimate tensile strength to the hardness was suggested.\\

[Microstructure]

This thesis leaves open numerous paths of investigation. An in-depth analysis of the variables influencing the residual stresses - such as machining and heat treatment conditions - could bring clarity regarding the reasons behind the surprising results of this work. A manufacturing process optimisation could also be carried out to minimise the residuals stresses in the parts, as their critical role was pointed out. Finally, other inherent issues of SLM should be addressed as fatigue strength.\\
%Consequently, although the principal goal of the thesis was not achieved, some 


%
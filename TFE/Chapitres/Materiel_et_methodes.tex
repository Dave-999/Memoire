\chapter{Materials and methods}
\label{Chap3}
Description expériences et machines
\section{Powder follow-up}

\subsection{Sieving}

\subsection{Grain size and distribution}

\subsection{Composition}

%\subsection{Drying}

\section{Process parameters}
The same direct metal printer (DMP) was used to fabricate all specimens throughout this work. It is a \textit{ProX DMP 200} printer, manufactured by \textit{3D Systems} (see figure \ref{fig:Printer}). It uses a laser with a theoritical maximal power of 300 [W] and wavelength $\lambda$ = 1070 [nm] \parencite{3D}. The actual maximal power is $P_{max}=273.6$  The maximal envelope capacity of the machine (W x D x H) is 140 x 140 x 125 [mm]. Its typical accuracy is +/- 50 [$\mu m$] for small parts and +/- 0.2\% for large parts. It allows for the set-up of a protection atmosphere. However, it does not integrate any heating feature for the build bed.\\

\begin{figure}[th]
\centering
\includegraphics[scale=0.7]{Images/Printer}
\decoRule
\caption[ProX DMP 200 printer]{ProX DMP 200 printer (from the user's ProX DMP 200 general instructions document).}
\label{fig:Printer}
\end{figure}

In this thesis, argon was used as shielding gas. [Corrections, paramètres laser etc...]. Values for h and t were respectively set to 100 [$\mu m$] and 30 [$\mu m$]. The other process parameters were varied so as to optimize the properties of the built specimens. Educated guesses were made based on literature and previous works done at the UCL. The parameters used are resumed in table \ref{tab:param}. Batches were named in the format X200-\textit{yymmdd}. The prefix "X200" refers to the DMP used. It is followed by the date of printing (6 digits). Recycled powder was used for every batch except for X200-180222 and X200-180228. [Dual scan strategy] . Dimensions of the cubic and cylindrical specimens are noted in accordance with figure \ref{fig:cc}.\\

 %\begin{center}
\begin{table}[ht]
\noindent\makebox[\textwidth]{\begin{tabular}{|c|c|c |c |c|c|c|}

    \hline
  Batch name & Contour & Type & Dimensions [mm] &Specimen name & $\frac{P}{P_{max}} [-]$ & $v_s [\frac{mm}{s}]$\\
  \hline
  \hline
  X200-171024 & No & Cubic & L=10& 1 & 0.85 & 900\\
  & &   & & 2 &  & 1000\\
  & &   & & 3&  & 1059\\
  & &  & & 4&  & 1500\\
  & &  & & 5& 1 & 900\\
  & &  & & 6&  & 1059\\
  & & & & 7, 7a, 7b & 0.75 & 1200\\
  & & & & 8, 8a, 8b& & 900\\
\hline  
  X200-180109 & No&Cubic & L=10 & 7c,...7q (15 spec.)& 0.75 &1200\\
  & & & & 8c,...8q (15 spec.)  & & 900\\
\hline  
  X200-180222 & No & Cubic & L=10 &12& 0.75 & 1200\\
  & &  & &13 &  &\\
\hline  
  X200-180228  & Yes & Cylindrical & D=6, H=2 &1 & 0.75 & 1200\\
  & &  &  & 2&  & \\
  & &  &  & 3 &  & \\
\hline  
  X200180313 & Yes & Cylindrical & D=6, H=10&1& 0.75 & 1200\\
    & &    & &2 & &  \\
    & &  &D=12, H=10 &3& &  \\
    & & & &4 & & \\
\hline  
  X200-180319  & Yes & Cubic & L=10 & cub 1 & 0.75 & 1200\\
  & &  & & cub 2 & &\\
  & &  & & cub 3 & &\\
  & & & & cub 4 & &\\
  & & & & cub 5 & &\\
  & & &  L=5& TT??????? &  &\\ 
  & &  Cylindrical & D=6, H=10&cyl 1   &  & \\
    & &  & &cyl 2 & & \\
    & &  & D=12, H=10 &cyl 3 & & \\
    & &  &  &cyl 4  & & \\

    \hline
\end{tabular}}
\label{tab:param}
\caption[Process parameters used for the specimens manufacturing]{Process parameters used for the specimens manufacturing}
\end{table}
 %\end{center}
 
\begin{figure}[th]
\centering
\includegraphics[scale=0.58]{Images/cc}
\caption[Dimensions notations for (a) cubic specimens (b) cylindrical specimens]{Dimensions notations for (a) cubic specimens (b) cylindrical specimens}
\label{fig:cc}
\end{figure}

Insérer images des positions d'échantillons.


\section{Heat treatments}


\section{Characterisation}

\subsection{Density}

\subsubsection{...}

\subsection{Microscopy}

\subsubsection{Scanning electron microscope microscopy}

\subsubsection{Optical microscopy}

\subsection{Mechanical properties}

\subsubsection{Hardness test}

\subsubsection{Traction test}

\subsubsection{Fatigue}
